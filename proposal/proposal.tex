\documentclass{article}
\usepackage{amssymb,amsmath}
\usepackage [colorlinks = true,
            linkcolor = blue,
            urlcolor  = blue,
            citecolor = blue,
            anchorcolor = blue]{hyperref}
\usepackage{tikz}
\title{Raytracing Project}
\author{Jared Givens, Ahbi Sohal, Noah Krim}
\date{28 May, 2023}


\begin{document}
\maketitle
\textbf{Serial Implementation}
The program outputs color values of pixels with raytracing techniques. 
It begins by initalizing virtual geometry and a camera. Then it sends rays from the camera 
into the virtual space. 

For each pixel a series of uniformly random rays are 
cast to gather a color value. the color values are then averaged to achive anti 
aliassing. 

Each ray iterates over the scenes geometry to determine the nearest 
intersection. The nearest intersection's color is then multiplied against the 
rays running total color value. The ray then reflects or refracts based on the 
properties of the material that was intersected. The process of accumulating 
color and bouncing continues until the ray fails to collide with an object or 
reaches the maximum ray depth. If the ray fails to collide, the program adds the 
contribution of the skybox to complete the color.  If the ray reaches the 
maximum bounce depth the color is set to black.  

The serial prototype is only capable of outputing frames after several seconds.
This program is an excellent candidate for CUDA because the code of each ray can 
be run in parallel. Additionally the code for each ray contains many branches 
that could be removed with further linear algebra. 

Our team can complete this project because we completed a serial 
prototype in cpp this weekend: 

\href{https://github.com/JaredGivens/ECS158-Raytracing}{Repo}

Following the tutorial:

\href{https://raytracing.github.io/books/RayTracingInOneWeekend.html}{RayTracingInOneWeekend}

\end{document}.